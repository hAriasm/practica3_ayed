%package list
\documentclass{article}
\usepackage[top=3cm, bottom=3cm, outer=3cm, inner=3cm]{geometry}
\usepackage{graphicx}
\usepackage{url}
%\usepackage{cite}
\usepackage{hyperref}
\usepackage{array}
\usepackage{multicol}
\newcolumntype{x}[1]{>{\centering\arraybackslash\hspace{0pt}}p{#1}}
\usepackage{natbib}
\usepackage{pdfpages}
\usepackage{multirow}
\usepackage{float}
\usepackage[normalem]{ulem}
\useunder{\uline}{\ul}{}
\usepackage{svg}
\usepackage{amsmath}
\usepackage{hyperref}

%%%%%%%%%%%%%%%%%%%%%%%%%%%%%%%%%%%%%%%%%%%%%%%%%%%%%%%%%%%%%%%%%%%%%%%%%%%%
%%%%%%%%%%%%%%%%%%%%%%%%%%%%%%%%%%%%%%%%%%%%%%%%%%%%%%%%%%%%%%%%%%%%%%%%%%%%
\newcommand{\csemail}{vmachacaa@unsa.edu.pe}
\newcommand{\csdocente}{Vicente Machaca Arceda}
\newcommand{\cscurso}{Algoritmos y Estructura de Datos}
\newcommand{\csuniversidad}{Universidad Nacional de San Agustín}
\newcommand{\csescuela}{Maestría en Ciencias de la Computación}
\newcommand{\cspracnr}{02}
\newcommand{\cstema}{Estructura de datos}
%%%%%%%%%%%%%%%%%%%%%%%%%%%%%%%%%%%%%%%%%%%%%%%%%%%%%%%%%%%%%%%%%%%%%%%%%%%%
%%%%%%%%%%%%%%%%%%%%%%%%%%%%%%%%%%%%%%%%%%%%%%%%%%%%%%%%%%%%%%%%%%%%%%%%%%%%


\usepackage[english,spanish]{babel}
\usepackage[utf8]{inputenc}
\AtBeginDocument{\selectlanguage{spanish}}
\renewcommand{\figurename}{Figura}
\renewcommand{\refname}{Referencias}
\renewcommand{\tablename}{Tabla} %esto no funciona cuando se usa babel
\AtBeginDocument{%
	\renewcommand\tablename{Tabla}
}

\usepackage{fancyhdr}
\pagestyle{fancy}
\fancyhf{}
\setlength{\headheight}{30pt}
\renewcommand{\headrulewidth}{1pt}
\renewcommand{\footrulewidth}{1pt}
\fancyhead[L]{\raisebox{-0.2\height}{\includegraphics[width=3cm]{img/logo_unsa}}}
\fancyhead[C]{}
\fancyhead[R]{\fontsize{7}{7}\selectfont	\csuniversidad \\ \csescuela \\ \textbf{\cscurso} }
\fancyfoot[L]{Grupo N 02}
\fancyfoot[C]{\cscurso}
\fancyfoot[R]{Página \thepage}


\begin{document}
	
	\vspace*{10px}
	
	\begin{center}	
		\fontsize{17}{17} \textbf{ Práctica \cspracnr}
	\end{center}


	\begin{table}[h]
		\begin{tabular}{|x{4.7cm}|x{4.8cm}|x{4.8cm}|}
			\hline
			\textbf{DOCENTE} & \textbf{CARRERA}  & \textbf{CURSO}   \\
			\hline
			\csdocente & \csescuela & \cscurso    \\
			\hline
		\end{tabular}
	\end{table}	
	
	
	\begin{table}[h]
		\begin{tabular}{|x{4.7cm}|x{4.8cm}|x{4.8cm}|}
			\hline
			\textbf{PRÁCTICA} & \textbf{TEMA}  & \textbf{DURACIÓN}   \\
			\hline
			\cspracnr & \cstema & --   \\
			\hline
		\end{tabular}
	\end{table}
	
	\section{Integrantes}
        	\begin{itemize}
        		\item Grupo N 2
        		\item Integrantes:
        		\begin{itemize}
        			\item EDER ALONSO, AMPUERO ATAMARI
        			\item HOWARD FERNANDO, ARANZAMENDI MORALES
        			\item JOSE EDISON, PEREZ MAMANI
        			\item HENRRY IVAN, ARIAS MAMANI
        		\end{itemize}		
        	\end{itemize}
    \section{Repositorio GitHub}
           URL Github: \href{https://github.com/hAriasm/practica3_ayed}{Repositorio Práctica 3 Algoritmos y Estructura de Datos (AyED)}
	\section{Estructuras de Datos}
    \subsection{Quatree}
        \paragraph{}
        El árbol AVL recibe su nombre de las iniciales  de sus inventores, Georgii Adelson-Velskii y Yevgeniy Landis. Dieron a conocer esto mediante la publicación de un artículo en 1962, " Un algoritmo para organizar la información" (" Un algoritmo para organizar la información").



    \section{Conclusiones}
        \begin{itemize}
                 \item Un árbol AVL se mantiene ordenado, pero hay mas rotaciones en las inserciones que en las eliminaciones, su costo para buscar, eliminar, insertar  es de O(Log n).s
        \end{itemize}
    \section{Referencias}
  \begin{enumerate}
    \item González, A. H. (2013). Operaciones sobre árboles.
    \item	Weiss, M. A., Jorge tr Lozano Moreno, Andoni colab. téc Eguíluz,   Inés colab. téc Jacob. (1995). Estructuras de datos y algoritmos.
  \end{enumerate}

\end{document} 